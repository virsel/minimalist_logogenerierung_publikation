\newpage
\section{Fazit und Ausblick}\label{sec:fazit}

\subsection{Zusammenfassung der Arbeit}

Die vorliegende Arbeit untersuchte, wie ein \ac{KI}-Modell für konditionierte Bildgenerierung ressourceneffizient auf die Erstellung minimalistischer Logos optimiert werden kann. Durch die Kombination von \ac{LoRA}-basiertem Finetuning und ControlNet-gesteuerter Bildgenerierung wurde ein Prototyp entwickelt, der auf handelsüblicher Hardware (NVIDIA RTX 5080 mit 16 GB \acs{VRAM}) betrieben werden kann.

Die experimentelle Evaluation bestätigte die ersten drei aufgestellten Hypothesen eindeutig: (H1) ControlNet verbesserte die strukturelle Übereinstimmung dramatisch (SSIM-Score +32,4\,\%), (H2) das \ac{LoRA}-Feintuning steigerte die semantische Qualität (CLIP-Score +3,7\,\%) und visuelle Realitätstreue (FID-Reduktion -30,9\,\%), (H3) die systematische Hyperparameter-Analyse identifizierte die Lernrate als dominantesten Faktor mit optimalen Ergebnissen bei $lr=1e-4$ und ``extended''-Konfiguration. Die qualitative Evaluation (H4) offenbarte ein gemischtes Bild: Während die stilistische Analyse mit objektivem Fokus klare Vorteile bei der Einhaltung eines minimalistischen Stils – etwa durch die zuverlässige Umsetzung eines einfarbigen Hintergrunds – offenbarte, aber auch Schwächen bei Details zeigte, ergab die subjektive Nutzerbefragung mit 47 Teilnehmenden nur eine leichte, statistisch nicht signifikante Tendenz zugunsten des feinabgestimmten Modells (52,1\,\%, p=0,3808).

Die Arbeit liefert praktische Leitlinien für das parameter-effiziente Feintuning von Diffusionsmodellen in kreativen Domänen und demonstriert, dass hochwertige Ergebnisse auch ohne spezialisierte Recheninfrastruktur erzielbar sind.

\subsection{Wissenschaftlicher Beitrag und Limitationen}

Der wissenschaftliche Beitrag der Arbeit liegt darin, gezeigt zu haben, dass etablierte Richtwerte für Trainingsparameter aus der allgemeinen Bildgenerierungsforschung auch auf die spezifische Domäne des Logodesigns übertragbar sind, um gute Ergebnisse zu erzielen. Die Experimente bestätigten, dass eine hohe Lernrate von $1e-4$ für das \ac{LoRA}-Training, wie sie in der Literatur für \acs{PEFT}-Methoden im Gegensatz zum vollständigen Feintuning empfohlen wird \parencite{HU2021}, auch hier zu den besten Ergebnissen bei Bildqualität (\ac{FID}) und semantischer Kohärenz (\ac{CLIP}) führt. Im Gegensatz dazu zeigte sich, dass der \ac{LoRA}-Rang eine untergeordnete Rolle spielte. Entgegen der Annahme von \textcite[S. 10]{HU2021}, dass Ränge von 4 oder 8 einen „Sweetspot“ darstellen könnten, funktionierte in den Experimenten letztendlich ein Rang von 32 in Kombination mit der höchsten Lernrate am besten, um die Evaluierungsergebnisse zu optimieren. Weiterhin konnte gezeigt werden, dass die für das Training gewählte Datensatzgröße und Trainingsdauer, die sich an etablierten Empfehlungen von \textcite{cloneofsimo_lora}, \textcite{ruiz2023dreamboothfinetuningtexttoimage} und \textcite{ZHANG2023} orientierten, für gute Ergebnisse ausreichend waren.

Die qualitative Befragung mit 47 Teilnehmenden konnte trotz konsistenter Tendenzen keine statistisch signifikante Überlegenheit des feinabgestimmten Modells nachweisen (p=0,3808), was auf zu geringe wahrgenommene Unterschiede bei dieser Stichprobengröße hindeutet. Auf Grundlage dieser Erkenntnis wurde für einen kommerziellen Einsatz erkannt, dass eine umfangreichere und qualitativ hochwertigere Datengrundlage sowie ein entsprechend intensiveres und optimiertes Training notwendig sind, um die Modellqualität entscheidend zu verbessern. Weitere Limitationen bestehen in der Datenbasis (spezifische Stichprobe, systematische Caption-Qualitätsprobleme) und den hardwarebedingten Einschränkungen (Batch-Größe=8).

\subsection{Ausblick}

Gemäß der in dieser Arbeit angewandten \acs{TDSP}-Methodik wurden in Kapitel~\ref{sec:limitations_and_implications} bereits potenzielle Optimierungsschritte in Form von Feedbackschleifen dargelegt. Als zentraler Hebel für zukünftige Qualitätssteigerungen wurde dort insbesondere die Datenqualität identifiziert. Darüber hinaus stellt die Modellarchitektur generell einen entscheidenden Faktor für die Leistungsfähigkeit dar. Es ist anzunehmen, dass diese beiden Dimensionen den größten Einfluss auf die Leistungsfähigkeit spezialisierter generativer Modelle besitzen.

Die dynamische Weiterentwicklung im Bereich der generativen \ac{KI} lässt erwarten, dass der Aufwand für die Erstellung und Kuratierung hochqualitativer Trainingsdaten stetig abnehmen wird. Die Arbeiten von \textcite{Gangadharaiah2024} zeigen zudem, dass die ansteigende Qualität bei der Generierung hochqualitativer synthetischer Daten neue Möglichkeiten eröffnet, um Datensätze gezielt zu erweitern und zu verbessern. Fortschritte bei \acp{VLM} und anderen KI-gestützten Analysemethoden – ein Forschungsfeld, das \textcite{YANG20231000} umfassend beleuchten – werden eine präzisere und effizientere Bewertung der Datenqualität ermöglichen. Parallel dazu treiben Forschungseinrichtungen wie \textcite{DeepMind_Research} vielversprechende Entwicklungen im Bereich neuer Modellarchitekturen und Trainingstechniken voran, die die Leistungsfähigkeit weiter steigern werden. In Kombination dieser beiden Entwicklungsstränge dürfte die Erstellung spezialisierter Modelle, wie des hier vorgestellten Prototyps für Logodesign, zunehmend effektiver und ressourcenschonender werden.

Angesichts dieser Entwicklungen ist davon auszugehen, dass generative \ac{KI}-Modelle künftig eine immer wichtigere Rolle als Werkzeuge zur Arbeitserleichterung in kreativen und technischen Domänen einnehmen werden. Das Potenzial zur Einsparung von Ressourcen wie Zeit und Kosten bei gleichzeitig hoher Ergebnisqualität macht sie zu einem wertvollen Instrument für Designer, Entwickler und Unternehmen.

\subsection{Schlussbemerkung}

Diese Arbeit demonstriert, dass die Kombination von multimodaler Konditionierung und parameter-effizientem Feintuning eine effektive Strategie zur Spezialisierung von Diffusionsmodellen auf kreative Designaufgaben darstellt. Die systematische Hyperparameter-Analyse liefert praktische Leitlinien, die über die spezifische Domäne des Logodesigns hinaus relevant sind.

Die Ergebnisse unterstreichen das Potenzial generativer \ac{KI} als Werkzeug zur Unterstützung kreativer Prozesse – nicht als Ersatz menschlicher Expertise, sondern als Erweiterung, die neue Möglichkeiten der Exploration und Iteration eröffnet. Die bewusste Fokussierung auf Ressourceneffizienz trägt zur Demokratisierung des Zugangs zu fortgeschrittenen \ac{KI}-Technologien bei und erhöht die praktische Relevanz für eine breite Zielgruppe. Die identifizierten Limitationen bieten konkrete Ansatzpunkte für iterative Verbesserungen im Sinne der \acs{TDSP}-Methodik und weisen den Weg für zukünftige Forschungsarbeiten.


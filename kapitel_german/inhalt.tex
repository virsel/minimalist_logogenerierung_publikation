% Kapitelstruktur der Masterarbeit (Skeleton)

\section{Einleitung (6)}\label{sec:einleitung}
% Motivation, Problemstellung, Zielsetzung, Forschungsfrage, Beitrag, Aufbau
\subsection{Motivation und Problemstellung}
\subsection{Zielsetzung und Forschungsfragen}
\subsection{Vorgehensweise und Aufbau der Arbeit}

\section{Theoretische Grundlagen (12)}\label{sec:grundlagen}
% Designprinzipien, KI-Grundlagen, Multimodalität, rechtl./ethische Aspekte
\subsection{Minimalistisches Logodesign: Prinzipien und Kriterien}
\subsection{Generative KI-Modelle: Diffusion, ControlNet, CLIP}
\subsection{Multimodale Modellierung: Bild+Text Repräsentationen}
\subsection{Bewertungsmetriken und Qualitätskriterien}

\section{Methodik (14)}\label{sec:methodik}
% Forschungsdesign, Datensatz, Modellwahl, Feintuning-Protokoll
\subsection{Forschungsdesign und Hypothesen}
\subsection{Datenbasis und Aufbereitung}
\subsection{Modellauswahl und Systemarchitektur}
\subsection{Feintuning-Strategie und Hyperparameter}

\section{Implementierung (13)}\label{sec:implementierung}
% Pipeline, Preprocessing, Trainingsumgebung, Inferenz
\subsection{Pipeline-Design und Komponenten}
\subsection{Trainings- und Inferenzumgebung}
\subsection{Reproduzierbarkeit und Versionierung}

\section{Ergebnisse (14)}\label{sec:ergebnisse}
% Quant. und qual. Ergebnisse, Beispiele, Abbildungen, Tabellen
\subsection{Quantitative Evaluation}
\subsection{Qualitative Analyse und Fallbeispiele}
\subsection{Ablations- und Sensitivitätsanalysen}

\section{Diskussion (7)}\label{sec:diskussion}
% Interpretation, Limitationen, Implikationen
\subsection{Interpretation der Ergebnisse}
\subsection{Limitationen und Validität}
\subsection{Implikationen für Forschung und Praxis}

\section{Fazit und Ausblick (4)}\label{sec:fazit}
% Zusammenfassung und Ausblick
\subsection{Zusammenfassung der Arbeit}
\subsection{Ausblick und zukünftige Arbeiten}